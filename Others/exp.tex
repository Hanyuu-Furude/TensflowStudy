\documentclass[12pt, a4papper]{ctexart}
 \usepackage[utf8]{inputenc}
 \usepackage{comment}
 \usepackage[colorlinks,linkcolor = blue]{hyperref}
 \usepackage{amsmath}

% Title
\title{Hanyuu Document}
\author{Hanyuu Furude}
\date{\today}
\begin{document}
\begin{titlepage}
\maketitle
\end{titlepage}
\pagenumbering{arabic}
\tableofcontents
Object 01
Object 02
\begin{abstract}
This is a simple paragraph at the beginning of the document. A brief introduction about the main subject.  \\
这是文档开头的一个简单段落。关于主要主题的简要介绍。
\end{abstract}
First document. This is a simple example, with no extra parameters or packages included.\\
同样支持数学公式哦。\\
行内公式样式:$\sum_{i=1}^{n}sin\left(i\right)$\\
行间公式样式:$$\sum_{i=1}^{n}sin\left(i\right)$$\\
组合公式:
\begin{equation}
		\oint_{0}^{1}x^{3.2}
\end{equation}
\begin{equation}
	\frac{\sqrt[3.14]{x^2}}{\lceil a \rceil}
\end{equation}\\
\href{https://hanyuufurude.github.io/Others/Latex/}{有关公式语法,请参阅}\\
\href{mailto:Furude_Hanyuu@outlook.com}{有问题?联系我}\\
\setcounter{page}{page number}
% Comments
\begin{comment}
This text won't show up in the compiled pdf this is just a multi-line comment. Useful to, for instance, comment out slow-rendering while working on the draft.
\end{comment}
\end{document}
